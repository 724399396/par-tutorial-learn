\Subsection{conc-higher}{Higher-level concurrency abstractions}

The preceding sections covered the basic interfaces for writing
concurrent code in Haskell.  These are enough for simple tasks, but
for larger and more complex tasks we need to raise the level of
abstraction.

Earlier in \secref{mvars} we introduced the @Async@ interface for
performing operations asynchronously and waiting for the results.  In
this section we will be revisiting that interface and expanding it
with some more sophisticated functionality.  In particular, we will

\begin{itemize}
\item provide a way to create an @Async@ that is automatically
  cancelled if its parent dies,
\item provide a way to wait for an @Async@ that propagates exceptions
  from the @Async@ to the parent,
\item and we will provide various ways to wait for multiple @Async@s simultaneously.
\end{itemize}

What we are aiming for is the ability to build \emph{trees of
  threads}, such that when a process dies for whatever reason, two
things happen: any children it has are automatically terminated, and
its parent is informed.  This is part of the ``let it crash''
philosophy promoted by the designers of Erlang: the idea is that
rather than trying to build elaborate error-recovery machinery, we
should simply program the normal case and let every abnormal situation
cause the thread (or process in the case of Erlang) to crash.  The
crash is typically caught by a supervisor process that knows how to
restart the system in a known good state.

In order to implement the ``let it crash'' philosophy, we need to be
able to spawn threads that will inform their parent when they exit.
In Erlang this is done by \emph{linking} processes together, a
mechanism that is provided natively by the Erlang runtime.  In
Haskell, we can build equivalent functionality using the mechanisms
introduced earlier: threads, asynchronous exceptions, and STM.

